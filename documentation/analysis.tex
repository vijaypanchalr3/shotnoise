
This is code for where I made my tools for data analysis. This tools presents with me file opening, file reading, taking mean over single frequency data, sorting my data with deviation method, plotting single data points form points etc. all the data and code is at following github link \href{https://github.com/vijaypanchalr3/shotnoise}{https://github.com/vijaypanchalr3/shotnoise}. 

\emph{tools.py}

\begin{minted}[breaklines, breakanywhere, linenos, xleftmargin=20pt]{python}
import csv
import re
from numpy import array,mean,abs,split,vstack,argsort,column_stack

__all__=[
    "files"
    ]

class files:
    """
    PARAMETER: filename as Relative path to __file__
    RETURN: nil
    FUNCTION: read files named filename, write other files with data
    """
    def __init__(self,filename:str,datatype=float) -> None:
        self.datatype = datatype
        self.filename = filename

        with open(filename, 'r',) as newfile:
            
            self.fobject = list(csv.reader(newfile))
            
            # omit first member
            try:
                datatype(self.fobject[0][0])
                self.header = None
            except ValueError:
                self.header = self.fobject.pop(0)
                
            self.length = sum(1 for row in self.fobject)
            
    def file_add(self,nameaddition:str="extra"):
        _finalfile = re.split("\/",self.filename)
        finalfile = re.split("\.", _finalfile[len(_finalfile)-1])
        self.finalfile = "/".join(str(_finalfile[i]) for i in range(len(_finalfile)-1))+"/"+finalfile[0]+nameaddition+finalfile[1]
    
    def write_another(self,data:list)-> None:
        with open(self.finalfile, 'wxs', newline="") as cleandata:
            writer = csv.writer(cleandata)
            writer.writerow(data)
    
    def get_mean(self):
        data = array(self.fobject)
        try:
            first_array,second_array = split(data,2,axis=1)
        except IndexError:
            print("from get_mean()::empty array from data")
            
        extra_array,_first_array,_second_array = [],[],[]
        count = 0
        _first_array.append(first_array[count][0])
        for i in range(0,len(data)):
            if first_array[i][0]==_first_array[count]:
                extra_array.append(second_array[i][0])
            else:
                _second_array.append(mean(array(extra_array,dtype=float)))
                count+=1
                _first_array.append(first_array[i][0])
                extra_array=[]

            if i==len(first_array)-1:
                _second_array.append(mean(array(extra_array,dtype=float)))

        return vstack((array(_first_array,dtype=float),array(_second_array,dtype=float))).T
                
    def sort_on_deviation(self, points=5):
        data = array(self.fobject, dtype=float)
        filtered_data=[]
        temp_erray = []
        index = float(data[0][0])
        count = 0
        for datapoint in data:
            count+=1
            if datapoint[0]==index:
                temp_erray.append(float(datapoint[1]))
            else:
                nlist = array(temp_erray,dtype=float)
                m = mean(nlist)
                sorted_deviation = argsort(abs(nlist - m))
                filtered_nlist = nlist[sorted_deviation[:points]]
                indexonelist = array([index]*points)
                final_list= column_stack((indexonelist,filtered_nlist))
                for member in final_list:
                    filtered_data.append(member)
                index = float(datapoint[0])
                temp_erray = []

            if count == len(data)-1:
                nlist = array(temp_erray,dtype=float)
                m = mean(nlist)
                sorted_deviation = argsort(abs(nlist - m))
                filtered_nlist = nlist[sorted_deviation[:points]]
                indexonelist = array([index]*points)
                final_list= column_stack((indexonelist,filtered_nlist))
                for member in final_list:
                    filtered_data.append(member)
        return array(filtered_data)
    
    def shady_plot(self, color="Blues"):
        """
        
        """
        data = array(self.fobject,dtype=float)
        
        # [here] can be better memort handling !
        common_array = [[]]
        index = float(data[0][0])
        count = 0
        for datapoint in data:
            if float(datapoint[0])==index:
                try:
                    common_array[count].append([float(datapoint[0]),float(datapoint[1])])
                    count+=1
                except IndexError:
                    common_array.append([])
                    common_array[count].append([float(datapoint[0]),float(datapoint[1])])
                    count+=1
            else:
                count=0
                index = float(datapoint[0])
        
        
        

        delete = []
        common_array.pop()
        for i in range(len(common_array)-1):    
            if len(common_array[len(common_array)-1])<len(common_array[i]):
                common_array[i].pop()
        
        common_array = array(common_array)
        
        from matplotlib import cm

        colormap = cm.get_cmap(color, len(common_array))

        return common_array,colormap


    def point_mean(self,data):
        freq = []
        vo = []
        freq.append(float(data[0][0]))
        vo.append(float(data[0][1]))
        count = 1
        for i in range(1,len(data)):
            lenth = len(freq)
            if float(data[i][0])==freq[lenth-1]:
                vo[lenth-1]+=float(data[i][1])
                count+=1
            else:
                vo[lenth-1]=vo[lenth-1]/count
                count=1
                freq.append(float(data[i][0]))
                vo.append(float(data[i][1]))
        freq.pop()
        vo.pop()
        return array(freq,dtype=float),array(vo,dtype=float)



 

if __name__=="__main__":
    print("No Error")
\end{minted}



Some analysis code,

\begin{minted}[breaklines, breakanywhere, linenos, xleftmargin=20pt]{python}
import tools

import matplotlib.pyplot as plt
import matplotlib as mpl
import matplotlib.font_manager as font_manager

mpl.rcParams['font.family']='serif'
cmfont = font_manager.FontProperties(fname=mpl.get_data_path() + 'cmr10.ttf')
mpl.rcParams['font.serif']=cmfont.get_name()
mpl.rcParams['mathtext.fontset']='cm'
mpl.rcParams['axes.unicode_minus']=False
leg_font = font_manager.FontProperties(size=12)
font = {'color':'black','size':12}

import numpy as np

from scipy.optimize import curve_fit




file1 = "../data/final2.csv"
file2 = "../data/file3002.csv"
file3 = "../data/filelow.csv"
file4 = "../data/file10001.csv"
file5 = "../data/file100001.csv"
file6 = "../data/file10002.csv"

ufile1 = "../data/file1k100us.csv"
ufile2 = "../data/file1k2100us.csv"
ufile3 = "../data/file1k3100us.csv"

file1_init = tools.files(file1)
file2_init = tools.files(file2)
file3_init = tools.files(file3)
file4_init = tools.files(file4)
file5_init = tools.files(file5)
file6_init = tools.files(file6)
# file7_init = tools.files(file7)
# file8_init = tools.files(file8)

ufile1_init = tools.files(ufile1)
ufile2_init = tools.files(ufile2)
ufile3_init = tools.files(ufile3)




diff_data, colormap = file3_init.shady_plot("plasma")
total_points=5
fig, ax = plt.subplots()
for i in range(2,len(diff_data),int(len(diff_data)/total_points)):
    ax.plot(diff_data[i][:,0],diff_data[i][:,1],color=colormap(i), label=f"run {i+1}",linewidth= .6)
    # ax.plot(diff_data[i][:,0],diff_data[i][:,1],'ro')
ax.plot(diff_data[0][:,0],diff_data[0][:,1],"r",label="FIRST RUN",linewidth =1.5)
# ax.set(ylim=(0.0,0.001))
ax.set_xlabel("frequency",fontdict=font)
ax.set_ylabel("R",fontdict=font)
# ax.set_title("RAW data: Noise for sub 1Hz frequency with TIME CONSTANT=100$ms$",fontdict=font)
ax.grid(which="both",axis="both",color="orange",alpha=0.4)
ax.legend(prop= leg_font)
plt.savefig("raw1100us.png",dpi=700)



diff_data, colormap = file5_init.shady_plot("plasma")
total_points=10
fig, ax = plt.subplots()
for i in range(0,len(diff_data),int(len(diff_data)/total_points)):
    ax.plot(diff_data[i][:,0],diff_data[i][:,1],color=colormap(i), label=f"data {i}",linewidth= .6)
# ax.set(ylim=(0.0,0.0002))
ax.legend()
ax.set_xlabel("frequency")
ax.set_ylabel("Vrms(Noise)")


diff_data, colormap = file6_init.shady_plot("plasma")
total_points=10
fig, ax = plt.subplots()
for i in range(0,len(diff_data),int(len(diff_data)/total_points)):
    ax.plot(diff_data[i][:,0],diff_data[i][:,1],color=colormap(i), label=f"data {i}",linewidth= .6)
# ax.set(ylim=(0.0,0.0002))
ax.legend()
ax.set_xlabel("frequency")
ax.set_ylabel("Vrms(Noise)")

diff_data, colormap = ufile1_init.shady_plot("plasma")
data = ufile1_init.get_mean()
total_points=5
fig, ax = plt.subplots()
for i in range(1,len(diff_data),int(len(diff_data)/total_points)):
    ax.plot(diff_data[i][:,0],diff_data[i][:,1],color=colormap(i), label=f"run {i}",linewidth= .6)
    # ax.plot(diff_data[i][:,0],diff_data[i][:,1],'ro')
ax.plot(data[:,0],data[:,1],"r",label="MEAN",linewidth =1.5)
# ax.set_xscale('log')
# ax.set(ylim=(0.0,0.001))
ax.set_xlabel("frequency",fontdict=font)
ax.set_ylabel("R",fontdict=font)
# ax.set_title("RAW data: Noise for sub 10k frequency with TIME CONSTANT=100$\mu s$",fontdict=font)
ax.grid(which="both",axis="both",color="orange",alpha=0.4)
ax.legend(prop= leg_font)
plt.savefig("raw1000100ms.png",dpi=700)


data = ufile1_init.sort_on_deviation(5)
fig, ax = plt.subplots()
ax.plot(diff_data[i][:,0],diff_data[i][:,1], label="",linewidth= .6)
# ax.plot(diff_data[i][:,0],diff_data[i][:,1],'ro')
# ax.set(ylim=(0.0,0.001))
ax.set_xlabel("frequency",fontdict=font)
ax.set_ylabel("$V_{rms}$(Noise)",fontdict=font)
ax.set_title("RAW data: zener diode data for sub 10k frequency with TIME CONSTANT=100$\mu s$",fontdict=font)
ax.grid(which="both",axis="both",color="orange",alpha=0.4)
ax.legend(prop= leg_font)
plt.savefig("mean1000100us.png",dpi=500)



\end{minted}
