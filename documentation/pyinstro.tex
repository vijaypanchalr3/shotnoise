
PyInstro is a package we made to communicate, control and data logging to any scientific instrument easily. The main work of it is giving utility for data logging and ease SCPI. Also, it does streamline instruments after extending it. It is just a cover for the PYVISA backend but it gives instrument specific tools. whole package in the following link,
This is code for just the GPIB connection which we used. (it is also extensible to the RS232, LAN and USB ). You can find PyInstro in following repositery, \href{https://github.com/vijaypanchalr3/pyinstro}{https://github.com/vijaypanchalr3/pyinstro}. 


%% \begin{table*}[hbt!]
\emph{GPIB.py}
\begin{minted}[breaklines,breakanywhere,linenos, xleftmargin=20pt]{python}
import pyvisa
import sys
from termcolor import cprint

class GPIB:
    def __init__(self) -> None:                                 
        try:                        # GPIB connection check
            cprint("-----------checking GPIB connections--------",color="yellow")
            resources = pyvisa.ResourceManager()
            interface = None
            resourceslist = resources.list_resources()
            cprint(resourceslist,'blue',attrs=['bold'])
            if resourceslist==():
                cprint("ERROR: please check GPIB connection", "red")
                sys.exit()
            else:
                while True:
                    try:
                        choise = int(input("please, choose your device from this list: "))-1
                        if choise>len(resourceslist):
                            TypeError
                        interface = resourceslist[choise]
                        cprint("-------------chose resource-----------------",color="green",attrs=["bold"])
                        cprint("-------following device is connected--------",color="green",attrs=["bold"])
                        cprint(interface)
                        break
                    except:
                        cprint("choose with interger and from following...","red")

            self.interface =  resources.open_resource(interface)
        except:
            cprint("ERROR in detecting GPIB, there must be problem with setup of pyvisa or there is no connection of gpib\n you should look either in pyvisa documentation or try for RS232 interface","red",attrs=['bold'])
            sys.exit()

    def ping(self)-> None:
        self.interface.write("*IDN?\n")

    def read(self)-> None:
        self.interface.read()

    def reset(self)-> None:
        self.interface.write("*RST\n")

    def clear_status(self)-> None:
        self.interface.write("*CLS\n")

    def close(self)->None:
        self.interface.close()

    def std_event(self)->None:
        pass

\end{minted}
%% \end{table*}

This is SR830 device commands,

\emph{SR830.py}
\begin{minted}[breaklines,breakanywhere,linenos, xleftmargin=20pt]{python}
from pyinstro.utils import sysarg
from pyinstro.utils import datafile

new_instance = sysarg.CLI()

if new_instance.get_connection()=="GPIB":
    from pyinstro.interfaces import gpib
    
    class SR830(gpib.GPIB):
        def __init__(self) -> None:
            super().__init__()

            file_init = datafile.Get_File(new_instance.get_file())
            
            self.get_levels = new_instance.get_levels
            self.get_partitions = new_instance.get_partitions
            self.writerow = file_init.writerow
            self.longwriterow = file_init.longwriterow
            self.fmin = new_instance.get_fmin
            self.fmax = new_instance.get_fmax
            self.freq = new_instance.get_freq

        def local_defaults(self)-> None:
            pass

        def local_arguments(self)-> None:
            new_instance.argparser.add_argument('-fl','--fmin', metavar='', type=float, default=4545, help="give lower limit for reference frequency")
            new_instance.argparser.add_argument('-fr','--freq', metavar='', type=float, default=7888, help="give reference frequency")
            new_instance.argparser.add_argument('-fh','--fmax', metavar='', type=float, default=1, help="give upper limit for reference frequency")
    
        def set_frequency(self, value, errdelay = 3) -> None:
            """change reference frequency"""
            self.interface.write("FREQ "+"{:.4E}".format(value))
            pass

        def autogain(self)->None:
            self.interface.write("AGAN")

        def set_phase(self,value) -> None:
            self.interface.write("PHAS "+str(value))
            pass

        def time_constant(self,choise) -> None:
            self.interface.write("OFLT "+str(choise))
            pass

        def sensitivity(self,choise) -> None:
            self.interface.write("SENS "+str(choise))
            pass

        def set_sample_rate(self, choise)->None:
            self.interface.write("SRAT "+str(choise))

        def start_data_acquision(self) -> None:
            self.interface.write("STRT")
            pass

        def pause_data_acquision(self) -> None:
            self.interface.write("PAUS")
            pass

        def reset_data_acquision(self) -> None:
            self.interface.write("REST")
            pass

        def get_data(self) -> None:
            pass

        def get_data_explicitly(self, data_variable=3, errdelay=3):
            """
            two params, give resource object and the second params is parameter to variable read,
            default to data_variable = 3 which is equievalent to reading R.
            as SR830manual, 
            data_variable = 1 => X,
            data_variable = 2 => Y,
            data_variable = 3 => R,
            data_variable = 4 => phase
            """
            return self.interface.query("OUTP? "+str(data_variable))

        
else:

    from pyinstro.interfaces import rs232
    
    class SR830(rs232.RS232):
\end{minted}

This is some utilities to ease control of scientific instruments,

FIlewrite \ simple data logger: \emph{getfile.py}

\begin{minted}[breaklines,breakanywhere,linenos, xleftmargin=20pt]{python}
from pyinstro.utils import getpath



import csv
import os



class Get_File:
    """
    INFO: just to write file, must be CSV 
    """
    def __init__(self,file) -> None:
        _project_dir_path_abs = getpath.getpath()

        if os.path.exists(os.path.join(_project_dir_path_abs,"data")):
            _data_dir_path_abs = os.path.join(_project_dir_path_abs,"data")
        else:
            os.mkdir(os.path.join(_project_dir_path_abs,"data"))
            _data_dir_path_abs = os.path.join(_project_dir_path_abs,"data")
        
        if file=='default':
            file = "auto0.csv"
            count = 0
            while os.path.exists(os.path.join(_data_dir_path_abs,f"auto{count}.csv")):
                count+=1
                file = f"auto{count}.csv"

        file = os.path.join(_data_dir_path_abs,file)
        
        # i did not used re module down here
        if not ((file[len(file)-1]=='v')and(file[len(file)-2]=='s')and(file[len(file)-3]=='c')and(file[len(file)-4]=='.')):
            file = file+".csv"
        else:
            pass
        
        self.filepath = file
        self.firsttime = True

    def writerow(self, data)-> None:
        """
        open file one time ad write it
        """
        if self.firsttime:
            self.file = open(self.filepath,'w',newline='')
            self.writer = csv.writer(self.file)
            self.writer.writerow(data)
            self.firsttime = False
            print(self.filepath)
        else:
            self.writer.writerow(data)


    def longwriterow(self,data)->None:
        """
        for long data, i think it is suitable to write file each time open and close
        """
        with open(self.filepath,'a',newline="") as datafile:
            self.writer = csv.writer(datafile)
            self.writer.writerow(data)

\end{minted}


DEFAULT setting: \emph{defaults.py}

\begin{minted}[breaklines,breakanywhere,linenos, xleftmargin=20pt]{python}
from pyinstro.utils import getpath

import os
import configparser

class DefaultParams:
    """
    specify default parameters
    
    for more info:
    refer to SR830 manual for more info.
    """
    time_constant = 5
    sensitivity = 5
    # filter_slope = 

    baud_rate = 9600
    sample_rate = 10
    gpib_address = 1
    time_delay = 1

    connection = 1          # means GPIB, 1: GPIB, 2: RS232, 3: USB, 4: LAN
    connections = {1:"GPIB", 2:"RS232", 3:"USB", 4:"LAN"}

    fmin = 01E+3
    fmax = 01E+5
    
    partitions = 4
    levels = 4

    data= 3

    def __init__(self) -> None:
        self.defaults_params_list= [attr for attr in dir(self) if not callable(getattr(self, attr)) and not attr.startswith("__")]
        #defaults_params= dict(zip(defaults_params_list,list(" "*len(defaults_params_list))))
        
        self.target_path = getpath.getpath()
        config_file = os.path.join(self.target_path,"config.ini")

        print("checking config.ini file")
        
        if os.path.exists(config_file):
            config = configparser.ConfigParser()
            config.read(config_file)
            config_file_dict = config.defaults()
            if len(config_file_dict)==len(self.defaults_params_list):
                for keys in config_file_dict:
                    if (config_file_dict[keys].isspace() or not config_file_dict[keys]):
                        pass
                    else:
                        setattr(self, keys, config_file_dict[keys])
                        print(keys+":  "+config_file_dict[keys])
            else:
                for keys in self.defaults_params_list:
                    if not (keys in config_file_dict):
                        with open(config_file, "w") as _conf_file: 
                            config.set("DEFAULT",keys," ")
                            config.write(_conf_file) 
                    else:
                        if (config_file_dict[keys].isspace() or config_file_dict[keys] ==""):
                            pass
                        else:
                            setattr(self, keys, config_file_dict[keys])
                            print(keys+":  "+config_file_dict[keys])
                            
        else:
            pass

    def makeconfig(self):
        config_file =os.path.join(self.target_path,"config.ini")
        config = configparser.ConfigParser()
        if os.path.exists(config_file):
            config.read(config_file)
            config_file_dict = config.defaults()
            if len(config_file_dict)==len(self.defaults_params_list):
                pass
            else:
                for keys in self.defaults_params_list:
                    if not (keys in config_file_dict):
                        with open(config_file, "w") as _conf_file: 
                            config.set("DEFAULT",keys," ")
                            config.write(_conf_file)
                            
                    else:
                        pass
                print("I had appened to full option to config file!")
        else:
            with open(config_file,'w') as config_file:
                config_file.write("[DEFAULT]\n")
                for params in self.defaults_params_list:
                    config_file.write(params+" = \n")
            print("I had made config file in present directory !")

\end{minted}



You can use this package following way. I made simple sampler for data acqusion.

\begin{minted}[breaklines, breakanywhere, linenos, xleftmargin=20pt]{python}
from pyinstro import SR830

import numpy
import sys
import time

class sampler:
    """
    function:   very simpler data logger for SR830
                write file, as what given from terminal or auto{number}.csv in present directory under data directory.
    limitation: too much hard coded
    """
    def __init__(self) -> None:
        self.device = SR830()
        time.sleep(2)
        self.device.ping()
        time.sleep(0.5)
        print(self.device.read())
        time.sleep(2)
        self.device.longwriterow(["Frequency", "RinV"])

    def discrete_range(self,minimum,maximum,step):
        
        self.device.set_frequency(minimum)
        time.sleep(.8)
        self.device.autogain()
        input()
        for freq in range(minimum,maximum,step):
            self.device.set_frequency(freq)
            print(freq)
            time.sleep(.5)
            self.device.autogain()
            time.sleep(1)
            for j in range(1):
                for i in range(50):
                    data= float(self.device.get_data_explicitly(3))
                    self.device.longwriterow([freq,data])
                    # print(freq,data)
                    time.sleep(0.1)

    def partition_loop(self,minimum, maximum,partitions,timedelay=0.2):
        # time.sleep(2)
        frange = numpy.linspace(minimum, maximum,partitions)
        count = 1
        for freq in frange:
            self.device.set_frequency(freq)
            time.sleep(timedelay)
            for i in range(100):
                data = float(self.device.get_data_explicitly(3))
                self.device.longwriterow([freq,data])
                print(data)
                input()
                time.sleep(timedelay)
                if count==50:
                    input("check setup and press enter")
                    count = 1
                else:
                    count+=1

    

if __name__=="__main__":
    x = sampler()
    x.discrete_range(1018,2018,50)
    sys.exit()
\end{minted}
            
